\setbeamerfont{block title}{size=\scriptsize}

\section{Advanced package aspects}

\subsection{Licensing report}

\begin{frame}[fragile]{Licensing report: introduction}
  \begin{itemize}
  \item A key aspect of embedded Linux systems is {\bf license
      compliance}.
  \item Embedded Linux systems integrate together a number of
    open-source components, each distributed under its own license.
  \item The different open-source licenses may have {\bf different
      requirements}, that must be met before the product using the
    embedded Linux system starts shipping.
  \item Buildroot helps in this license compliance process by offering
    the possibility of generating a number of {\bf license-related
      information} from the list of selected packages.
  \item Generated using:
\begin{block}{}
\begin{verbatim}
$ make legal-info
\end{verbatim}
\end{block}
  \end{itemize}
\end{frame}

\begin{frame}{Licensing report: contents of {\tt legal-info}}
  \begin{itemize}
  \item \code{sources/}, all the source files that are redistributable
    (tarballs, patches, etc.)
  \item \code{buildroot.config}, the Buildroot \code{.config} file
  \item \code{host-manifest.csv}, a CSV file with the list of {\em
      host packages}, their version, license, etc.
  \item \code{host-licenses/<pkg>/}, the full license text of all {\em
      host packages}, per package
  \item \code{host-licenses.txt}, the full license text of all {\em
      host packages}, in a single file
  \item \code{licenses.txt}, the full license text of all {\em target
      packages}, in a single file
  \item \code{README}
  \item \code{licenses/}, the full license text of all {\em target
      packages}, per package
  \item \code{manifest.csv}, a CSV file with the list of {\em target
      packages}, their version, license, etc.
  \end{itemize}
\end{frame}

\begin{frame}{Including licensing information in packages}
  \begin{itemize}
  \item \code{<pkg>_LICENSE}
    \begin{itemize}
    \item Comma-separated {\bf list of license(s)} under which the
      package is distributed.
    \item Free form string, but should use the license codes from
      \url{https://spdx.org/licenses/}
    \item Can indicate which part is under which license (programs,
      tests, libraries, etc.)
    \end{itemize}
  \item \code{<pkg>_LICENSE_FILES}
    \begin{itemize}
    \item Space-separated {\bf list of file paths} from the package
      source code containing the license text and copyright
      information
    \item Paths relative to the package top-level source directory
    \end{itemize}
  \item \code{<pkg>_REDISTRIBUTE}
    \begin{itemize}
    \item Boolean indicating whether the package source code should be
      redistributed or not (part of the \code{legal-info} output)
    \item Defaults to \code{YES}, can be overriden to \code{NO}
    \end{itemize}
  \end{itemize}
\end{frame}

\begin{frame}[fragile]{Licensing information examples}
  \begin{block}{linux.mk}
\begin{minted}{make}
LINUX_LICENSE = GPLv2
LINUX_LICENSE_FILES = COPYING
\end{minted}
  \end{block}

\begin{block}{acl.mk}
\begin{minted}{make}
ACL_LICENSE = GPLv2+ (programs), LGPLv2.1+ (libraries)
ACL_LICENSE_FILES = doc/COPYING doc/COPYING.LGPL
\end{minted}
\end{block}

\begin{block}{owl-linux.mk}
\begin{minted}{make}
OWL_LINUX_LICENSE = PROPRIETARY
OWL_LINUX_LICENSE_FILES = LICENSE
OWL_LINUX_REDISTRIBUTE = NO
\end{minted}
\end{block}

\end{frame}

\subsection{User, permission and device tables}

\begin{frame}[fragile]{Package-specific users}

\begin{itemize}

\item The default skeleton in \code{system/skeleton/} has a number of
  default users/groups.
\item Packages can define their own custom users/groups using the
  \code{<pkg>_USERS} variable:
{\footnotesize
  \begin{block}{}
\begin{verbatim}
define <pkg>_USERS
        username uid group gid password home shell groups comment
endef
\end{verbatim}
  \end{block}}
\item Examples:
  \begin{block}{}
    \begin{minted}[fontsize=\footnotesize]{make}
define AVAHI_USERS
        avahi -1 avahi -1 * - - -
endef
\end{minted}
\end{block}

\begin{block}{}
  \begin{minted}[fontsize=\footnotesize]{make}
define MYSQL_USERS
        mysql -1 nogroup -1 * /var/mysql - - MySQL daemon
endef
  \end{minted}
\end{block}
\end{itemize}

\end{frame}

\begin{frame}[fragile]{File permissions and owernship}

\begin{itemize}

\item By default, before creating the root filesystem images,
  Buildroot changes the ownership of all files to \code{0:0}, i.e
  \code{root:root}
\item Permissions are preserved as is, but since the build is executed
  as non-root, it is not possible to install setuid applications.
\item A default set of permissions for certain files or directories is
  defined in \code{system/device_table.txt}.
\item The \code{<pkg>_PERMISSIONS} variable allows packages to define
  special ownership and permissions for files and directories:
{\small
  \begin{block}{}
\begin{verbatim}
define <pkg>_PERMISSIONS
name type mode uid gid major minor start inc count
endef
\end{verbatim}
  \end{block}}
\item The \code{major}, \code{minor}, \code{start}, \code{inc} and
  \code{count} fields are not used.

\end{itemize}

\end{frame}

\begin{frame}[fragile]{File permissions and owership: examples}

  \begin{itemize}
  \item \code{sudo} needs to be installed {\em setuid root}:
  \begin{block}{}
    \begin{minted}[fontsize=\small]{make}
define SUDO_PERMISSIONS
        /usr/bin/sudo f 4755 0 0 - - - - -
endef
\end{minted}
\end{block}

\item \code{/var/lib/nginx} needs to be owned by \code{www-data},
  which has UID/GID \code{33} defined in the skeleton:

\begin{block}{}
  \begin{minted}[fontsize=\small]{make}
define NGINX_PERMISSIONS
        /var/lib/nginx d 755 33 33 - - - - -
endef
\end{minted}
\end{block}

\end{itemize}

\end{frame}

\begin{frame}[fragile]{Devices}
  \begin{itemize}
  \item Definining devices only applies when the chosen \code{/dev}
    management strategy is {\em Static using a device table}. In other
    cases, {\em device files} are created dynamically.
  \item A default set of {\em device files} is described in
    \code{system/device_table_dev.txt} and created by Buildroot in the
    root filesystem images.
  \item When packages need some additional custom devices, they can
    use the \code{<pkg>_DEVICES} variable:
{\small
  \begin{block}{}
\begin{verbatim}
define <pkg>_DEVICES
name type mode uid gid major minor start inc count
endef
\end{verbatim}
  \end{block}}
\item Becoming less useful, since most people are using a dynamic
  \code{/dev} nowadays.
  \end{itemize}
\end{frame}

\begin{frame}[fragile]{Devices: example}

    \begin{block}{xenomai.mk}
      \begin{minted}{make}
define XENOMAI_DEVICES
/dev/rtheap  c  666  0  0  10  254  0  0  -
/dev/rtscope c  666  0  0  10  253  0  0  -
/dev/rtp     c  666  0  0  150 0    0  1  32
endef
      \end{minted}
    \end{block}

\end{frame}

\subsection{Init scripts and systemd unit files}

\begin{frame}{Init scripts, systemd unit files}

\begin{itemize}

\item Buildroot supports several main init systems: {\em sysvinit},
  {\em Busybox} and {\em systemd}

\item When packages want to install a program to be started at boot
  time, they need to install either a startup script ({\em
    sysvinit}/{\em Busybox}) or a {\em systemd service} file.

\item They can do so with the \code{<pkg>_INSTALL_INIT_SYSV} and
  \code{<pkg>_INSTALL_INIT_SYSTEMD} variables, which contain a list of
  shell commands.

\item Buildroot will execute either the \code{<pkg>_INSTALL_INIT_SYSV}
  or the \code{<pkg>_INSTALL_INIT_SYSTEMD} commands of all enabled
  packages depending on the selected init system.

\end{itemize}

\end{frame}

\begin{frame}[fragile]{Init scripts, systemd unit files: example}

\begin{block}{bind.mk}
  \begin{minted}[fontsize=\scriptsize]{make}
define BIND_INSTALL_INIT_SYSV
  $(INSTALL) -m 0755 -D package/bind/S81named \
     $(TARGET_DIR)/etc/init.d/S81named
endef

define BIND_INSTALL_INIT_SYSTEMD
  $(INSTALL) -D -m 644 package/bind/named.service \
     $(TARGET_DIR)/usr/lib/systemd/system/named.service

  mkdir -p $(TARGET_DIR)/etc/systemd/system/multi-user.target.wants

  ln -sf /usr/lib/systemd/system/named.service \
    $(TARGET_DIR)/etc/systemd/system/multi-user.target.wants/named.service
endef
  \end{minted}
\end{block}

\end{frame}

\subsection{Config scripts}

\begin{frame}{Config scripts: introduction}
  \begin{itemize}
  \item Libraries not using \code{pkg-config} often install a {\bf
      small shell script} that allows applications to query the
    compiler and linker flags to use the library.
  \item Examples: \code{curl-config}, \code{freetype-config}, etc.
  \item Such scripts will:
    \begin{itemize}
    \item generally return results that are {\bf not appropriate for
        cross-compilation}
    \item be used by other cross-compiled Buildroot packages that use
      those libraries
    \end{itemize}
  \item By listing such scripts in the \code{<pkg>_CONFIG_SCRIPTS}
    variable, Buildroot will {\bf adapt the prefix, header and library
      paths} to make them suitable for cross-compilation.
  \item Paths in \code{<pkg>_CONFIG_SCRIPTS} are relative to
    \code{$(STAGING_DIR)/usr/bin}.
  \end{itemize}
\end{frame}

\begin{frame}[fragile]{Config scripts: examples}

  \begin{block}{libpng.mk}
    \begin{minted}[fontsize=\footnotesize]{make}
LIBPNG_CONFIG_SCRIPTS = \
        libpng$(LIBPNG_SERIES)-config libpng-config
\end{minted}
\end{block}

\begin{block}{imagemagick.mk}
  \begin{minted}[fontsize=\footnotesize]{make}
IMAGEMAGICK_CONFIG_SCRIPTS = \
        $(addsuffix -config,Magick MagickCore MagickWand Wand)

ifeq ($(BR2_INSTALL_LIBSTDCPP)$(BR2_USE_WCHAR),yy)
IMAGEMAGICK_CONFIG_SCRIPTS += Magick++-config
endif
\end{minted}
\end{block}

\end{frame}

\begin{frame}[fragile]{Config scripts: effect}

\begin{block}{Without \code{<pkg>_CONFIG_SCRIPTS}}
{\footnotesize
  \begin{verbatim}
$ ./output/staging/usr/bin/libpng-config --cflags --ldflags
-I/usr/include/libpng16
-L/usr/lib -lpng16
  \end{verbatim}}
\end{block}

\begin{block}{With \code{<pkg>_CONFIG_SCRIPTS}}
{\tiny
  \begin{verbatim}
$ ./output/staging/usr/bin/libpng-config --cflags --ldflags
-I.../buildroot/output/host/usr/arm-buildroot-linux-uclibcgnueabi/sysroot/usr/include/libpng16
-L.../buildroot/output/host/usr/arm-buildroot-linux-uclibcgnueabi/sysroot/usr/lib -lpng16
  \end{verbatim}}
\end{block}

\end{frame}

\subsection{Hooks}

\begin{frame}{Hooks: principle (1)}
  \begin{itemize}
  \item Buildroot {\em package infrastructure} often implement a
    default behavior for certain steps:
    \begin{itemize}
    \item \code{generic-package} implements for all packages the
      download, extract and patch steps
    \item Other infrastructures such as \code{autotools-package} or
      \code{cmake-package} also implement the configure, build and
      installations steps
    \end{itemize}
  \item In some situations, the package may want to do {\bf additional
      actions} before or after one these steps.
  \item The {\bf hook} mechanism allows packages to add such custom
    actions.
  \end{itemize}
\end{frame}

\begin{frame}{Hooks: principle (2)}
  \begin{itemize}
  \item There are {\bf pre} and {\bf post} hooks available for all
    steps of the package compilation process:
    \begin{itemize}
    \item download, extract, rsync, patch, configure, build, install,
      install staging, install target, install images, legal info
    \item \code{<pkg>_(PRE|POST)_<step>_HOOKS}
    \item Example: \code{CMAKE_POST_INSTALL_TARGET_HOOKS},
      \code{CVS_POST_PATCH_HOOKS}, \code{BINUTILS_PRE_PATCH_HOOKS}
    \end{itemize}
  \item Hook variables contain a list of make macros to call at the
    appropriate time.
    \begin{itemize}
    \item Use \code{+=} to register an additional hook to a hook point
    \end{itemize}
  \item Those make macros contain a list of commands to execute.
  \end{itemize}
\end{frame}

\begin{frame}[fragile]{Hooks: examples}

\begin{block}{libungif.mk: remove unneeded binaries}
  \begin{minted}[fontsize=\scriptsize]{make}
define LIBUNGIF_BINS_CLEANUP
        rm -f $(addprefix $(TARGET_DIR)/usr/bin/,$(LIBUNGIF_BINS))
endef

LIBUNGIF_POST_INSTALL_TARGET_HOOKS += LIBUNGIF_BINS_CLEANUP
  \end{minted}
\end{block}

\begin{block}{vsftpd.mk: adjust configuration}
  \begin{minted}[fontsize=\scriptsize]{make}
define VSFTPD_ENABLE_SSL
        $(SED) 's/.*VSF_BUILD_SSL/#define VSF_BUILD_SSL/' \
                $(@D)/builddefs.h
endef

ifeq ($(BR2_PACKAGE_OPENSSL),y)
VSFTPD_DEPENDENCIES += openssl
VSFTPD_LIBS += -lssl -lcrypto
VSFTPD_POST_CONFIGURE_HOOKS += VSFTPD_ENABLE_SSL
endif
  \end{minted}
\end{block}

\end{frame}

\subsection{Overriding commands}

\begin{frame}{Overriding commands: principle}

  \begin{itemize}
  \item In other situations, a package may want to completely {\bf
      override} the default implementation of a step provided by a
    package infrastructure.

  \item A package infrastructure will in fact only implement a given
    step {\bf if not already defined by a package}.

  \item So defining \code{<pkg>_EXTRACT_CMDS} or
    \code{<pkg>_BUILDS_CMDS} in your package \code{.mk} file will
    override the package infrastructure implementation (if any).

  \end{itemize}

\end{frame}

\begin{frame}[fragile]{Overriding commands: examples}

\begin{block}{jquery: source code is only one file}
\begin{minted}[fontsize=\scriptsize]{make}
JQUERY_SITE = http://code.jquery.com
JQUERY_SOURCE = jquery-$(JQUERY_VERSION).min.js

define JQUERY_EXTRACT_CMDS
        cp $(DL_DIR)/$(JQUERY_SOURCE) $(@D)
endef
\end{minted}
\end{block}

\begin{block}{tftpd: install only what's needed}
\begin{minted}[fontsize=\scriptsize]{make}
define TFTPD_INSTALL_TARGET_CMDS
        $(INSTALL) -D $(@D)/tftp/tftp $(TARGET_DIR)/usr/bin/tftp
        $(INSTALL) -D $(@D)/tftpd/tftpd $(TARGET_DIR)/usr/sbin/tftpd
endef

$(eval $(autotools-package))
\end{minted}
\end{block}

\end{frame}

\subsection{Legacy handling}

\begin{frame}{Legacy handling: {\tt Config.in.legacy}}
  \begin{itemize}
  \item When a \code{Config.in} option is removed, the corresponding
    value in the \code{.config} is silently removed.
  \item Due to this, when users upgrade Buildroot, they generally
    don't know that an option they were using has been removed.
  \item Buildroot therefore adds the removed config option to
    \code{Config.in.legacy} with a description of what has
    happened.
  \item If any of these legacy options is enabled then Buildroot
    refuses to build.
  \end{itemize}
\end{frame}

\subsection{Virtual packages}

\begin{frame}{Virtual packages}
  \begin{itemize}
  \item There are situations where different packages provide an
    implementation of the same interface
  \item The most useful example is OpenGL
    \begin{itemize}
    \item OpenGL is an API
    \item Each HW vendor typically provides its own OpenGL
      implementation, each packaged as separate Buildroot packages
    \end{itemize}
  \item Packages using the OpenGL interface do not want to know which
    implementation they are using: they are simply using the OpenGL
    API
  \item The mechanism of {\em virtual packages} in Buildroot allows to
    solve this situation.
    \begin{itemize}
    \item \code{libgles} is a virtual package offering the OpenGL ES API
    \item Eight packages are {\em providers} of the OpenGL ES API:
      \code{gpu-amd-bin-mx51}, \code{gpu-viv-bin-mx6q}, \code{mesa3d},
      \code{nvidia-driver}, \code{nvidia-tegra23-binaries},
      \code{rpi-userland}, \code{sunxi-mali}, \code{ti-gfx}
    \end{itemize}
  \end{itemize}
\end{frame}

\begin{frame}[fragile]{Virtual package definition: Config.in}

\begin{block}{libgles/Config.in}
{\small
\begin{verbatim}
config BR2_PACKAGE_HAS_LIBGLES
        bool

config BR2_PACKAGE_PROVIDES_LIBGLES
        depends on BR2_PACKAGE_HAS_LIBGLES
        string
\end{verbatim}}
\end{block}

\begin{itemize}
\item \code{BR2_PACKAGE_HAS_LIBGLES} is a hidden boolean
  \begin{itemize}
  \item Packages needing OpenGL ES will \code{depends on} it.
  \item Packages providing OpenGL ES will \code{select} it.
  \end{itemize}
\item \code{BR2_PACKAGE_PROVIDES_LIBGLES} is a hidden string
  \begin{itemize}
  \item Packages providing OpenGL ES will define their name as the
    variable value
  \item The \code{libgles} package will have a build dependency on
    this provider package.
  \end{itemize}
\end{itemize}

\end{frame}

\begin{frame}[fragile]{Virtual package definition: {\tt .mk}}

\begin{block}{libgles/libgles.mk}
\begin{minted}{make}
$(eval $(virtual-package))
\end{minted}
\end{block}

\begin{itemize}

\item Nothing to do: the \code{virtual-package} infrastructure takes
  care of everything, using the \code{BR2_PACKAGE_HAS_<name>} and
  \code{BR2_PACKAGE_PROVIDES_<name>} options.

\end{itemize}

\end{frame}

\begin{frame}[fragile]{Virtual package provider}

\begin{block}{sunxi-mali/Config.in}
{\small
\begin{verbatim}
config BR2_PACKAGE_SUNXI_MALI
        bool "sunxi-mali"
        select BR2_PACKAGE_HAS_LIBEGL
        select BR2_PACKAGE_HAS_LIBGLES

config BR2_PACKAGE_PROVIDES_LIBGLES
        default "sunxi-mali"
\end{verbatim}}
\end{block}

\begin{block}{sunxi-mali/sunxi-mali.mk}
\begin{minted}{make}
[...]
SUNXI_MALI_PROVIDES = libegl libgles
[...]
\end{minted}
\end{block}

\begin{itemize}
\item The variable \code{<pkg>_PROVIDES} is only used to detect if two
  providers for the same virtual package are enabled.
\end{itemize}

\end{frame}

\begin{frame}[fragile]{Virtual package user}

  \begin{block}{qt5/qt5base/Config.in}
{\small
\begin{verbatim}
config BR2_PACKAGE_QT5BASE_OPENGL_ES2
        bool "OpenGL ES 2.0+"
        depends on BR2_PACKAGE_HAS_LIBGLES
        help
          Use OpenGL ES 2.0 and later versions.
\end{verbatim}}
\end{block}

\begin{block}{qt5/qt5base/qt5base.mk}
\begin{minted}[fontsize=\small]{make}
ifeq ($(BR2_PACKAGE_QT5BASE_OPENGL_DESKTOP),y)
QT5BASE_CONFIGURE_OPTS += -opengl desktop
QT5BASE_DEPENDENCIES   += libgl
else ifeq ($(BR2_PACKAGE_QT5BASE_OPENGL_ES2),y)
QT5BASE_CONFIGURE_OPTS += -opengl es2
QT5BASE_DEPENDENCIES   += libgles
else
QT5BASE_CONFIGURE_OPTS += -no-opengl
endif
\end{minted}
\end{block}

\end{frame}

\setuplabframe
{Advanced packages}
{
  \begin{itemize}
  \item Create a \code{BR2_EXTERNAL} tree
  \item Package an application with a mandatory dependency and an
    optional dependency
  \item Package a library, hosted on Github
  \item Use {\em hooks} to tweak packages
  \item Add a patch to a package
  \end{itemize}
}

\subsection{Basics}

\begin{frame}
  \frametitle{Whole Android Stack}
  \begin{center}
    \includegraphics[height=0.85\textheight]{slides/android-application-basics/android-stack-apps.pdf}
  \end{center}
\end{frame}

\begin{frame}
  \frametitle{Android applications}
  \begin{itemize}
  \item Android applications are written mostly in Java using Google's
    SDK
  \item Applications are bundled into an Android PacKage (\code{.apk}
    files) which are archives containing the compiled code, data and resources for the
    application, so applications are completely self-contained
  \item You can install applications either through a market (Google
    Play Store, Amazon Appstore, F-Droid, etc) or manually (through
    ADB or a file manager)
  \item Of course, everything we have seen so far is mostly here to
    provide a nice and unified environment to application developers
  \end{itemize}
\end{frame}

\begin{frame}
  \frametitle{Applications Security}
  \begin{itemize}
  \item Once installed, applications live in their own sandbox,
    isolated from the rest of the system
  \item The system assigns a Linux user to every application, so that
    every application has its own user/group
  \item It uses this UID and files permissions to allow  the application
    to access only its own files
  \item Each process has its own instance of Dalvik, so code is
    running isolated from other applications
  \item By default, each application runs in its own process, which
    will be started/killed during system life
  \item Android uses the \emph{principle of least privilege}. Each
    application by default has only access to what it requires to work.
  \item However, you can request extra permissions, make several
    applications run in the same process, or with the same UID, etc.
  \end{itemize}
\end{frame}

\begin{frame}
  \frametitle{Applications Components}
  \begin{itemize}
  \item Components are the basic blocks of each application
  \item You can see them as entry points for the system in the application
  \item There is four types of components:
    \begin{itemize}
    \item Activities
    \item Broadcast Receivers
    \item Content Providers
    \item Services
    \end{itemize}
  \item Every application can start any component, even located in
    other applications. This allows to share components easily, and
    have very little duplication. However, for security reasons, you
    start it through an Intent and not directly
  \item When an application requests a component, the system starts
    the process for this application, instantiates the needed class and
    runs that component. We can see that there is no single point of
    entry in an application like \code{main()}
  \end{itemize}
\end{frame}

\begin{frame}
  \frametitle{Application Manifest}
  \begin{itemize}
  \item To declare the components present in your application, you
    have to write a XML file, \code{AndroidManifest.xml}
  \item This file is used to:
    \begin{itemize}
    \item Declare available components
    \item Declare which permissions these components need
    \item Revision of the API needed
    \item Declare hardware features needed
    \item Libraries required by the components
    \end{itemize}
  \end{itemize}
\end{frame}

\begin{frame}[fragile]
  \frametitle{Manifest HelloWorld}
\begin{minted}[fontsize=\scriptsize]{xml}
<?xml version="1.0" encoding="utf-8"?>
<manifest package="com.example.android">
    <application>
        <activity android:name=".ExampleActivity"
                  android:label="@string/example_label">
            <intent-filter>
                <action android:name="android.intent.action.MAIN"/>
                <category android:name="android.intent.category.LAUNCHER"/>
            </intent-filter>
        </activity>
        <uses-library android:name="com.example.android.pl" />
    </application>
</manifest>
\end{minted}
\end{frame}

\begin{frame}
  \frametitle{NDK}
  \begin{itemize}
  \item Google also provides a NDK to allow developers to write native
    code
  \item While the code is not run by Dalvik, the security guarantees
    are still there
  \item Allows to write faster code or to port existing C code to
    Android more easily
  \item Since Gingerbread, you can even code a whole application
    without writing a single line of Java
  \item It is still packaged in an apk, with a manifest, etc.
  \item However, there are some drawbacks, the main one being that you
    can't access the resources mechanism available from Java
  \end{itemize}
\end{frame}

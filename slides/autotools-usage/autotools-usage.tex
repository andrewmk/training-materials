\section{Autotools usage}

\begin{frame}{Why do we need {\em autotools}?}
  \begin{itemize}
  \item {\bf Portability} accross Unix systems, architectures, Linux
    distributions
    \begin{itemize}
    \item Some C functions do not exist everywhere, or have different
      names or prototypes, can behave differently
    \item Header files can be organized differently
    \item All libraries may not be available everywhere
    \end{itemize}
  \item {\bf Standardized} build procedure
    \begin{itemize}
    \item Standard options
    \item Standard environment variables
    \item Standard behavior
    \end{itemize}
  \end{itemize}
\end{frame}

\begin{frame}{Alternatives to {\em autotools}}
  \begin{itemize}
  \item Regular {\em Makefiles}
    \begin{itemize}
    \item Not very portable
    \item No configuration tests, or options
    \item Hard to take into account all dependencies (e.g dependencies
      on header files)
    \item No standardized behavior
    \end{itemize}
  \item CMake
    \begin{itemize}
    \item A more modern build system
    \item One language, instead of several for {\em autotools}
    \item More straightforward to use and understand
    \item A lot less widely used than {\em autotools}, but growing in
      popularity
    \item Also generates Makefiles, like {\em autotools}
    \end{itemize}
  \end{itemize}
\end{frame}

\begin{frame}{Using {\em autotools} based packages}
  \begin{itemize}
  \item The basic steps to build an {\em autotools} based software
    component are:
    \begin{enumerate}
    \item {\bf Configuration}\\
      \code{./configure}\\
      Will look at the available build environment, verify required
      dependencies, generate \code{Makefile}s and a \code{config.h}
    \item {\bf Compilation}\\
      \code{make}\\
      Actually builds the software component, using the generated
      Makefiles.
    \item {\bf Installation}\\
      \code{make install}\\
      Installs what has been built.
    \end{enumerate}
  \end{itemize}
\end{frame}

\begin{frame}{What is {\tt configure} doing?}
  \begin{center}
    \includegraphics[width=\textwidth]{slides/autotools-usage/what-configure-does.pdf}
  \end{center}
\end{frame}

\begin{frame}{Standard Makefile targets}
  \begin{itemize}
  \item \code{all}, builds everything. The default target.
  \item \code{install}, installs everything that should be installed.
  \item \code{install-strip}, same as \code{install}, but then strips
    debugging symbols
  \item \code{uninstall}
  \item \code{clean}, remove what was built
  \item \code{distclean}, same as \code{clean}, but also removes the
    generated {\em autotools} files
  \item \code{check}, run the test suite
  \item \code{installcheck}, check the installation
  \item \code{dist}, create a tarball
  \end{itemize}
\end{frame}

\begin{frame}{Standard filesystem hierarchy}
  \begin{itemize}
  \item \code{prefix}, defaults to {\em /usr/local}
    \begin{itemize}
    \item \code{exec-prefix}, defaults to \code{prefix}
      \begin{itemize}
      \item \code{bindir}, for programs, defaults to \code{exec-prefix/}{\em bin}
      \item \code{libdir}, for libraries, defaults to \code{exec-prefix/}{\em lib}
      \end{itemize}
    \end{itemize}
  \item \code{includedir}, for headers, defaults to \code{prefix/}{\em include}
  \item \code{datarootdir}, defaults to \code{prefix/}{\em share}
    \begin{itemize}
    \item \code{datadir}, defaults to \code{datarootdir}
    \item \code{mandir}, for man pages, defaults to \code{datarootdir/}{\em man}
    \item \code{infodir}, for info documents, defaults to \code{datarootdir/}{\em info}
    \end{itemize}
  \item \code{sysconfdir}, for configuration files, defaults to
    \code{prefix/}{\em etc}
  \item \code{--<option>} available for each of them
    \begin{itemize}
    \item E.g: \code{./configure --prefix=~/sys/}
    \end{itemize}
  \end{itemize}
\end{frame}

\begin{frame}{Standard configuration variables}
  \begin{itemize}
  \item \code{CC}, C compiler command
  \item \code{CFLAGS}, C compiler flags
  \item \code{CXX}, C++ compiler command
  \item \code{CXXFLAGS}, C++ compiler flags
  \item \code{LDFLAGS}, linker flags
  \item \code{CPPFLAGS}, C/C++ preprocessor flags
  \item and many more, see \code{./configure --help}
  \item E.g: \code{./configure CC=arm-linux-gcc}
  \end{itemize}
\end{frame}

\begin{frame}{System types: build, host, target}
  \begin{itemize}
  \item {\em autotools} identify three {\bf system types}:
    \begin{itemize}
    \item {\bf build}, which is the system where the build takes place
    \item {\bf host}, which is the system where the execution of the
      compiled code will take place
    \item {\bf target}, which is the system for which the program will
      generate code. This is only used for compilers, assemblers,
      linkers, etc.
    \end{itemize}
  \item Corresponding \code{--build}, \code{--host} and
    \code{--target} {\em configure} options.
    \begin{itemize}
    \item They are all automatically {\em guessed} to the current
      machine by default
    \item \code{--build}, generally does not need to be changed
    \item \code{--host}, must be overriden to do cross-compilation
    \item \code{--target}, needs to be overriden if needed (to
      generate a cross-compiler, for example)
    \end{itemize}
  \item Arguments to these options are {\em configuration names}, also
    called {\em system tuples}
  \end{itemize}
\end{frame}

\begin{frame}{System type: {\em configuration names}}
  \begin{itemize}
  \item A string identifying a combination of architecture, operating
    system, ABI and C library
  \item General format: \code{<arch>-<vendor>-<kernel>-<operating_system>}
    \begin{itemize}
    \item \code{<arch>} is the type of processor, i.e \code{arm},
      \code{i686}, etc.
    \item \code{<vendor>} is a free form string, which can be omitted
    \item \code{<kernel>} is always \code{linux} when working with
      Linux systems, or \code{none} for bare metal systems
    \item \code{<operating_system>} generally identifies the C library
      and ABI, i.e \code{gnu}, \code{gnueabi}, \code{eabi},
      \code{gnueabihf}, \code{uclibcgnueabihf}
    \end{itemize}
  \item Also often used as the {\em prefix} for cross-compilation
    tools.
  \item Examples
    \begin{itemize}
    \item \code{x86_64-amd-linux-gnu}
    \item \code{powerpc-mentor-linux-gnu}
    \item \code{armeb-linux-gnueabihf}
    \item \code{i486-linux-musl}
    \end{itemize}
  \end{itemize}
\end{frame}

\begin{frame}[fragile]{System type: native compilation example}

\begin{block}{}
{\small
\begin{verbatim}
$ ./configure
[...]
checking build system type... x86_64-unknown-linux-gnu
checking host system type... x86_64-unknown-linux-gnu
checking for gcc... gcc
[...]
checking how to run the C preprocessor... gcc -E
[...]
\end{verbatim}}
\end{block}

\end{frame}

\begin{frame}{Cross-compilation}
  \begin{itemize}
  \item By default, {\em autotools} will guess the {\bf host} machine
    as being the current machine
  \item To cross-compile, it must be overriden by passing the
    \code{--host} option with the appropriate {\em configuration name}
  \item By default, {\em autotools} will try to use the
    cross-compilation tools that use the {\em configuration name} as
    their prefix.
  \item If not, the variables \code{CC}, \code{CXX}, \code{LD},
    \code{AR}, etc. can be used to point to the cross-compilation
    tools.
  \end{itemize}
\end{frame}

\begin{frame}[fragile]{System type: cross compilation example}

\begin{block}{}
{\scriptsize
\begin{verbatim}
$ which arm-linux-gnueabihf-gcc
/usr/bin/arm-linux-gnueabihf-gcc
$ ./configure --host=arm-linux-gnueabihf
[...]
checking build system type... x86_64-unknown-linux-gnu
checking host system type... arm-unknown-linux-gnueabihf
checking for arm-linux-gnueabihf-gcc... arm-linux-gnueabihf-gcc
[...]
checking how to run the C preprocessor... arm-linux-gnueabihf-gcc -E
[...]
\end{verbatim}}
\end{block}

\end{frame}

\begin{frame}{Out of tree build}
  \begin{itemize}
  \item {\em autotools} support out of tree compilation by default
  \item Consists in doing the build in a directory separate from the
    source directory
  \item Allows to:
    \begin{itemize}
    \item Build different configurations without having to rebuild
      from scratch each time.
    \item Do not clutter the source directory with build related files
    \end{itemize}
  \item To use out of tree compilation, simply run the configure
    script from another empty directory
    \begin{itemize}
    \item This directory will become the build directory
    \end{itemize}
  \end{itemize}
\end{frame}

\begin{frame}[fragile]{Out of tree build: example}

  \begin{block}{}
{\scriptsize
\begin{verbatim}
strace-4.9 $ ls
configure configure.ac Makefile.am system.c NEWS
AUTHORS   COPYING     file.c       ioprio.c config.h
strace-4.9 $ mkdir ../strace-build-x86 ../strace-build-arm
strace-4.9 $ cd ../strace-build-x86
strace-build-x86 $ ../strace-4.9/configure
[...]
strace-build-x86 $ make
[...]
strace-build-x86 $ cd ../strace-build-arm
strace-build-arm $ ../strace-4.9/configure --host=arm-linux-gnueabihf
[...]
strace-build-arm $ make
[...]
\end{verbatim}}
\end{block}

\end{frame}

\begin{frame}[fragile]{Diverted installation with DESTDIR}

  \begin{itemize}

  \item By default, \code{make install} installs to the directories
    given in \code{--prefix} and related options.

  \item In some situations, it is useful to {\em divert} the
    installation to another directory
    \begin{itemize}
    \item Cross-compilation, where the build machine is not the
      machine where applications will be executed.
    \item Packaging, where the installation needs to be done in a
      temporary directory.
    \end{itemize}

  \item Achieved using the \code{DESTDIR} variable.
  \end{itemize}

  \begin{block}{}
{\tiny
\begin{verbatim}
strace-4.9 $ make DESTDIR=/tmp/test install
[...]
strace-4.9 $ find  /tmp/test/ -type f
/tmp/test/usr/local/share/man/man1/strace.1
/tmp/test/usr/local/bin/strace-log-merge
/tmp/test/usr/local/bin/strace-graph
/tmp/test/usr/local/bin/strace
\end{verbatim}}
\end{block}

\end{frame}

\begin{frame}{{\tt --prefix} or {\tt DESTDIR} ?}
  \begin{itemize}
  \item \code{--prefix} and \code{DESTDIR} are often misunderstood
  \item \code{--prefix} is the location where the programs/libraries
    will be placed when executed on the {\em host machine}
  \item \code{DESTDIR} is a way of temporarily diverting the
    installation to a different location.

  \item For example, if you use \code{--prefix=/home/<foo>/sys/usr},
    then binaries/libraries will look for icons in
    \code{/home/<foo>/sys/usr/share/icons}
    \begin{itemize}
    \item Good for native installation in \code{/home/<foo>/sys}
    \item {\bf Bad} for cross-compilation where the binaries will
      ultimately be in \code{/usr}
    \end{itemize}

  \end{itemize}
\end{frame}

\begin{frame}[fragile]{{\tt --prefix} or {\tt DESTDIR} use cases}
  \begin{itemize}
  \item Native compilation, install system-wide in \code{/usr}
    \begin{block}{}
{\tiny
\begin{verbatim}
$ ./configure --prefix=/usr
$ make
$ sudo make install
\end{verbatim}}
\end{block}
\item Native compilation, install in a user-specific directory:
  \begin{block}{}
{\tiny
\begin{verbatim}
$ ./configure --prefix=/home/<foo>/sys/
$ make
$ make install
\end{verbatim}}
\end{block}
\item Cross-compilation, install in \code{/usr}, diverted to a
  temporary directory where the system for the target is built
  \begin{block}{}
{\tiny
\begin{verbatim}
$ ./configure --prefix=/usr
$ make
$ make DESTDIR=/home/<foo>/target-rootfs/ install
\end{verbatim}}
\end{block}

\end{itemize}

\end{frame}

\begin{frame}{Analyzing issues}
  \begin{itemize}
  \item \code{autoconf} keeps a log of all the tests it runs in a file
    called \code{config.log}
  \item Very useful for analysis of \code{autoconf} issues
  \item It contains several sections: {\em Platform}, {\em Core
      tests}, {\em Running config.status}, {\em Cache variables}, {\em
      Output variables}, {\em confdefs.h}
  \item The end of the {\em Core tests} section is usually the most
    interesting part
    \begin{itemize}
    \item This is where you would get more details about the reason of
      the {\em configure} script failure
    \end{itemize}
  \end{itemize}
\end{frame}

\begin{frame}[fragile]{{\tt config.log} example}

\begin{block}{}
{\tiny
\begin{verbatim}
$ ./configure ...
[...]
checking for TIFFFlushData in -ltiff34... no
configure: WARNING: *** TIFF loader will not be built (TIFF library not found) ***
configure: error: 
*** Checks for TIFF loader failed. You can build without it by passing
*** --without-libtiff to configure but some programs using GTK+ may
*** not work properly

$ cat config.log
[...]
configure:18177: .../usr/bin/x86_64-linux-gcc -std=gnu99 -o conftest -D_LARGEFILE_SOURCE
   -D_LARGEFILE64_SOURCE -D_FILE_OFFSET_BITS=64   -Os  -static -Wall -D_LARGEFILE_SOURCE
    -D_LARGEFILE64_SOURCE -D_FILE_OFFSET_BITS=64 -DG_DISABLE_SINGLE_INCLUDES  -static
    conftest.c -ltiff34 -ljpeg -lz -lm  >&5
.../host/opt/ext-toolchain/bin/../lib/gcc/x86_64-buildroot-linux-uclibc/4.8.4/../../../../
    x86_64-buildroot-linux-uclibc/bin/ld: cannot find -ltiff34
.../host/opt/ext-toolchain/bin/../lib/gcc/x86_64-buildroot-linux-uclibc/4.8.4/../../../../
   x86_64-buildroot-linux-uclibc/bin/ld: cannot find -ljpeg
collect2: error: ld returned 1 exit status
configure:18177: $? = 1
configure: failed program was:
[...]
configure:18186: result: no
configure:18199: WARNING: *** TIFF loader will not be built (TIFF library not found) ***
configure:18210: error: 
*** Checks for TIFF loader failed. You can build without it by passing
*** --without-libtiff to configure but some programs using GTK+ may
*** not work properly
\end{verbatim}}
\end{block}

\end{frame}

\begin{frame}{autotools: {\em autoconf} and {\em automake}}
  \begin{itemize}
  \item The \code{configure} script is a shell script generated from
    \code{configure.ac} by a program called \code{autoconf}
    \begin{itemize}
    \item \code{configure.ac} used to be named \code{configure.in} but
      this name is now deprecated
    \item written in shell script, augmented with numerous {\em m4}
      macros
    \end{itemize}
  \item The \code{Makefile.in} are generated from \code{Makefile.am}
    files by a program called \code{automake}
    \begin{itemize}
    \item Uses special \code{make} variables that are expanded in
      standard \code{make} constructs
    \end{itemize}
  \item Some auxilliary tools like \code{autoheader} or \code{aclocal}
    are also used
    \begin{itemize}
    \item \code{autoheader} is responsible for generated the {\em
        configuration header} template, \code{config.h.in}
    \end{itemize}
  \item Generated files (\code{configure}, \code{Makefile.in},
    \code{Makefile}) should not be modified.
    \begin{itemize}
    \item Reading them is also very difficult. Read the real source
      instead!
    \end{itemize}
  \end{itemize}
\end{frame}

\begin{frame}{Distribution}
  \begin{itemize}
  \item In general:
    \begin{itemize}
    \item When a software is published as a {\em tarball}, the
      \code{configure} script and \code{Makefile.in} files are already
      generated and part of the tarball.
    \item When a software is published through {\em version control
        system}, only the real sources \code{configure.ac} and
      \code{Makefile.am} are available.
    \end{itemize}
  \item There are some exceptions (like tarballs not having
    pre-generated \code{configure}/\code{Makefile.in})
  \item Do not version control generated files!
  \end{itemize}
\end{frame}

\begin{frame}{Regenerating {\em autotools} files: {\tt autoreconf}}
  \begin{itemize}
  \item To generate all the files used by {\em autotools}, you could
    call \code{automake}, \code{autoconf}, \code{aclocal},
    \code{autoheader}, etc. manually.
    \begin{itemize}
    \item But it is not very easy and efficient.
    \end{itemize}
  \item A tool called \code{autoreconf} automates this process
    \begin{itemize}
    \item Useful option: \code{-i} or \code{--install}, to ask
      \code{autoreconf} to copy missing auxiliary files
    \end{itemize}
  \item Always use \code{autoreconf}!
  \end{itemize}
\end{frame}

\begin{frame}[fragile]{{\tt autoreconf} example}

\begin{block}{}
{\tiny
\begin{verbatim}
$ find . -type f
./src/main.c
./Makefile.am
./configure.ac

$ autoreconf -i
configure.ac:4: installing './compile'
configure.ac:3: installing './install-sh'
configure.ac:3: installing './missing'
Makefile.am: installing './depcomp'

$ find . -type f
./install-sh
./src/main.c
./config.h.in
./configure
./missing
./depcomp
./aclocal.m4
./Makefile.am
./autom4te.cache/traces.0
./autom4te.cache/output.1
./autom4te.cache/output.0
./autom4te.cache/requests
./autom4te.cache/traces.1
./compile
./Makefile.in
./configure.ac
\end{verbatim}}
\end{block}

\end{frame}

\begin{frame}{Overall organization}

  \begin{center}
    \includegraphics[width=\textwidth]{slides/autotools-usage/autoreconf.pdf}
  \end{center}

\end{frame}

\setuplabframe
{Usage of existing {\em autotools} projects}
{
  \begin{itemize}
  \item First build of an {\em autotools} package
  \item Out of tree build and cross-compilation
  \item Overriding cache variables
  \end{itemize}
}

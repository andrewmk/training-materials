\section{Writing recipes - going further}

\subsection{Splitting packages}

\begin{frame}
  \frametitle{Benefits}
  \begin{itemize}
    \item Packages can be split.
    \item Useful when a single remote repository provides multiple
      binaries or libraries.
    \item The list of packages to provide is defined by the
      \code{PACKAGES} variable.
  \end{itemize}
\end{frame}

\begin{frame}[fragile]
  \frametitle{Example}
  \begin{itemize}
    \item The \code{kexec tools} provides \code{kexec} and \code{kdump}:
    \begin{block}{}
    \begin{minted}{c}
require kexec-tools.inc
export LDFLAGS = "-L${STAGING_LIBDIR}"
EXTRA_OECONF = " --with-zlib=yes"

SRC_URI[md5sum] = "b9f2a3ba0ba9c78625ee7a50532500d8"
SRC_URI[sha256sum] = "..."

PACKAGES =+ "kexec kdump"

FILES_kexec = "${sbindir}/kexec"
FILES_kdump = "${sbindir}/kdump"
    \end{minted}
    \end{block}
  \end{itemize}
\end{frame}

\subsection{Packages features}

\begin{frame}
  \frametitle{Benefits}
  \begin{itemize}
    \item Features can be built depending on the needs.
    \item This allows to avoid compiling all features in a software
      component when only a few are required.
    \item A good example is \code{ConnMan}: Bluetooth support
      is built only if there is Bluetooth on the target.
    \item The \code{PACKAGECONFIG} variable is used to configure the
      build on a per feature granularity, for packages.
  \end{itemize}
\end{frame}

\begin{frame}
  \frametitle{\code{PACKAGECONFIG}}
  \begin{itemize}
    \item \code{PACKAGECONFIG} takes the list of features to enable.
    \item \code{PACKAGECONFIG[feature]} takes up to four arguments,
      separated by commas:
      \begin{enumerate}
        \item Argument used by the configuration task if the feature
          is enabled (\code{EXTRA_OECONF}).
        \item Argument added to \code{EXTRA_OECONF} if the feature is
          disabled.
        \item Additional build dependency (\code{DEPENDS}), if enabled.
        \item Additional runtime dependency (\code{RDEPENDS}), if enabled.
      \end{enumerate}
    \item Unused arguments can be omitted or left blank.
  \end{itemize}
\end{frame}

\begin{frame}[fragile]
  \frametitle{Example: from \code{ConnMan}}
  \begin{block}{}
  \begin{minted}{c}
PACKAGECONFIG ??= "wifi openvpn"

PACKAGECONFIG[wifi] = "--enable-wifi,            \
                       --disable-wifi,           \
                       wpa-supplicant"
PACKAGECONFIG[bluetooth] = "--enable-bluetooth,  \
                            --disable-bluetooth, \
                            bluez4"
PACKAGECONFIG[openvpn] = "--enable-openvpn,      \
                          --with-openvpn=...,    \
                          --disable-openvpn,     \
                          ,                      \
                          openvpn"
  \end{minted}
  \end{block}
\end{frame}

\subsection{Conditional features}

\begin{frame}[fragile]
  \frametitle{Conditional features}
  \begin{itemize}
    \item Some values can be set dynamically, thanks to a set of
      functions:
    \item \code{base_contains(variable, checkval, trueval,
      falseval, d)}: if \code{checkval} is found in
      \code{variable}, \code{trueval} is returned; otherwise
      \code{falseval} is used.
    \item Example:
      \begin{block}{}
      \begin{minted}[fontsize=\tiny]{c}
PACKAGECONFIG ??= "                                                       \
    ${@base_contains('DISTRO_FEATURES', 'wifi','wifi', '', d)}            \
    ${@base_contains('DISTRO_FEATURES', 'bluetooth','bluetooth', '', d)}  \
    ${@base_contains('DISTRO_FEATURES', '3g','3g', '', d)}"
      \end{minted}
      \end{block}
  \end{itemize}
\end{frame}

\subsection{Applying patches}

\begin{frame}
  \frametitle{Applying patches}
  \begin{itemize}
    \item Files ending in \code{.patch}, \code{.diff} or with the
      \code{apply=yes} parameter will be applied during the
      \code{do_patch} task.
    \item It is possible to select which tool will be used to apply
      the patches listed in \code{SRC_URI} variable with
      \code{PATCHTOOL}.
    \item By default, \code{PATCHTOOL = 'quilt'} in Poky.
    \item Possible values: \code{git}, \code{patch} and \code{quilt}.
    \item See \code{meta/classes/patch.bbclass}
  \end{itemize}
\end{frame}

\begin{frame}
  \frametitle{Resolving conflicts}
  \begin{itemize}
    \item The \code{PATCHRESOLVE} variable defines how to handle
    conflicts when applying patches.
    \item It has two valid values:
      \begin{itemize}
        \item \code{noop}: the build fails if a patch cannot be
          successfully applied.
        \item \code{user}: a shell is launched to resolve manually the
          conflicts.
      \end{itemize}
    \item By default, \code{PATCHRESOLVE = "noop"} in
      \code{meta-yocto}.
  \end{itemize}
\end{frame}

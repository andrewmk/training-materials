\setbeamerfont{block title}{size=\scriptsize}

\section{Autoconf advanced}

\subsection{Configuration header}

\begin{frame}[fragile]{Configuration header}
  \begin{itemize}
  \item Very often, C/C++ code needs to know the result of certain
    tests done by the \code{configure} script.
  \item A template C header file can be automatically generated by
    \code{autoheader}, generally named \code{config.h.in}
  \item The final header file is generated by \code{configure},
    generally named \code{config.h}
  \item Declared using \code{AC_CONFIG_HEADERS}
    \begin{block}{\code{configure.ac} extract}
{\scriptsize
\begin{verbatim}
AC_CONFIG_HEADERS([config.h])
\end{verbatim}}
    \end{block}

    \begin{block}{Example config.h}
{\scriptsize
\begin{verbatim}
/* Define if the complete vga libraries (vga, vgagl) are installed */
/* #undef HAVE_LIBVGA */

/* Define to 1 if you have the <limits.h> header file. */
#define HAVE_LIMITS_H 1
\end{verbatim}}
\end{block}
  \end{itemize}
\end{frame}

\begin{frame}[fragile]{{\tt AC\_DEFINE}}

  \begin{itemize}
  \item \code{AC_DEFINE} allows to create C definitions in the {\em
      configuration header}
  \item \code{AC_DEFINE (variable, value, [description])}
  \end{itemize}

  \begin{block}{\code{configure.ac}}
\begin{verbatim}
AC_DEFINE([FOOBAR], [42], [This is the foobar value])
\end{verbatim}
  \end{block}

  \begin{block}{Generated \code{config.h}}
\begin{minted}{c}
/* This is the foobar value */
#define FOOBAR 42
\end{minted}
\end{block}

\end{frame}

\subsection{Checking for functions, headers, libraries, etc.}

\begin{frame}{Checking for functions}
  \begin{itemize}
  \item You may need to check if certain functions are available
    and/or meet certain characteristics
  \item Family of \code{AC_FUNC_*} macros
    \begin{itemize}
    \item \code{AC_FUNC_FORK}, \code{AC_FUNC_GETLOADAVG}, \code{AC_FUNC_MALLOC}, etc.
    \item See {\em autoconf} manual for details
    \end{itemize}
  \item \code{AC_CHECK_FUNC[S]} to check for generic functions
    \begin{itemize}
    \item \code{AC_CHECK_FUNC (function, [action-if-found],
        [action-if-not-found])}
    \item \code{AC_CHECK_FUNCS (function..., [action-if-found],
        [action-if-not-found])}
    \item Results available
      \begin{itemize}
      \item \code{ac_cv_func_<function>} variable in
        \code{configure.ac}
      \item \code{HAVE_<FUNCTION>} defines in {\em configuration
          headers}
      \end{itemize}
    \end{itemize}
  \end{itemize}
\end{frame}

\begin{frame}[fragile]{{\tt AC\_CHECK\_FUNCS()} example}

\begin{block}{configure.ac}
{\tiny
\begin{verbatim}
AC_CHECK_FUNCS([printf foobar])
echo "ac_cv_func_printf: ${ac_cv_func_printf}"
echo "ac_cv_func_foobar: ${ac_cv_func_foobar}"
AC_CONFIG_HEADER([config.h])
\end{verbatim}}
\end{block}

\begin{block}{Execution of \code{./configure}}
\begin{minted}[fontsize=\tiny]{console}
$ ./configure
[...]
checking for printf... yes
checking for foobar... no
ac_cv_func_printf: yes
ac_cv_func_foobar: no
[...]
config.status: creating config.h
\end{minted}
\end{block}

\begin{block}{Generated \code{config.h}}
{\tiny
\begin{verbatim}
[...]
/* Define to 1 if you have the `foobar' function. */
/* #undef HAVE_FOOBAR */

/* Define to 1 if you have the `printf' function. */
#define HAVE_PRINTF 1
[...]
\end{verbatim}}
\end{block}

\end{frame}

\begin{frame}{Checking for headers}

  \begin{itemize}
  \item Much like \code{AC_FUNC_*} and \code{AC_CHECK_FUNC[S]}, but for headers
  \item Variety of \code{AC_HEADER_*} macros
    \begin{itemize}
    \item Check the autoconf manual for details
    \end{itemize}
  \item \code{AC_CHECK_HEADER[S]} for generic headers checking
    \begin{itemize}
    \item \code{AC_CHECK_HEADER (header-file, [action-if-found], [action-if-not-found], [includes])}
    \item \code{AC_CHECK_HEADERS (header-file..., [action-if-found], [action-if-not-found], [includes])}
    \item Results available in:
      \begin{itemize}
      \item \code{ac_cv_header_<header-file>} variable in
        \code{configure.ac}
      \item \code{HAVE_<HEADER>_H} define in \code{config.h}
      \end{itemize}
    \end{itemize}
  \end{itemize}
\end{frame}

\begin{frame}[fragile]{{\tt AC\_CHECK\_HEADERS} example}

\begin{block}{\code{configure.ac}}
{\scriptsize
\begin{verbatim}
[...]
AC_CHECK_HEADERS([spawn.h],
        [echo "Header spawn.h was found"; has_spawn=yes],
        [echo "Header spawn.h was not found"])
echo ${has_spawn}
[...]
\end{verbatim}}
\end{block}

\begin{block}{Execution of \code{./configure}}
\begin{minted}[fontsize=\scriptsize]{console}
$ ./configure
[...]
checking for spawn.h... yes
Header spawn.h was found
yes
[...]
\end{minted}
\end{block}

\end{frame}

\begin{frame}[fragile]{Checking for libraries}
  \begin{block}{}
\begin{verbatim}
AC_SEARCH_LIBS (function, search-libs,
                [action-if-found], [action-if-not-found],
                [other-libraries])
\end{verbatim}
  \end{block}

  \begin{itemize}
  \item Search for a library defining \code{function}, by linking a
    simple program calling \code{function}
  \item Tries first with no library, and then with the different
    libraries in \code{search-libs}, one after the other.
  \item If a library is found, \code{-llibrary} is prepended to the
    \code{LIBS} variable, so programs will be linked against
    it. \code{action-if-found} is executed.
  \item If not, \code{action-if-not-found} is executed
  \item \code{other-libraries} allows to pass additional
    \code{-l<foo>} arguments that may be needed for the link test to
    succeed.
  \item Result in \code{ac_cv_search_<function>}
  \end{itemize}
\end{frame}

\begin{frame}[fragile]{{\tt AC\_SEARCH\_LIBS} example}

\begin{block}{\code{configure.ac}}
{\small
\begin{verbatim}
AC_SEARCH_LIBS(mvwaddstr, [ncurses cursesX curses])
\end{verbatim}}
\end{block}

\begin{block}{Execution of \code{./configure}}
\begin{minted}[fontsize=\small]{console}
$ ./configure
[...]
checking for library containing mvwaddstr... -lncurses
[...]
$ grep ac_cv_search_mvwaddstr config.log
ac_cv_search_mvwaddstr=-lncurses
\end{minted}
\end{block}

\begin{block}{Compilation}
\begin{minted}[fontsize=\small]{console}
$ make
[...]
gcc  -g -O2   -o hello main.o common.o  -lncurses
[...]
gcc  -g -O2   -o test test.o common.o  -lncurses
\end{minted}
\end{block}

\end{frame}

\begin{frame}{Other checks}
  \begin{itemize}
  \item {\bf Programs} with \code{AC_CHECK_PROGS}
    \begin{itemize}
    \item \code{AC_CHECK_PROGS(PERL, [perl5 perl])}
    \end{itemize}
  \item {\bf Declarations} with \code{AC_CHECK_DECLS}
  \item {\bf Structure members} with \code{AC_CHECK_MEMBERS}
  \item {\bf Types} with \code{AC_CHECK_TYPES}
    \begin{itemize}
    \item \code{AC_CHECK_TYPES(int8_t)}
    \end{itemize}
  \item See the {\em autoconf} manual for details
  \end{itemize}
\end{frame}

\subsection{Custom tests}

\begin{frame}{Writing new tests}
  \begin{itemize}
  \item You can create your own tests by pre-processing, compiling or
    linking small test programs:
    \begin{itemize}
    \item Pre-processing test\\
      {\small \code{AC_PREPROC_IFELSE (input, [action-if-true], [action-if-false])}}
    \item Compiling test\\
      {\small \code{AC_COMPILE_IFELSE (input, [action-if-true], [action-if-false])}}
    \item Link test\\
      {\small \code{AC_LINK_IFELSE (input, [action-if-true], [action-if-false])}}
    \end{itemize}
  \item Input should be formatted with \code{AC_LANG_SOURCE} or
    \code{AC_LANG_PROGRAM}
  \item Runtime tests can also be created
    \begin{itemize}
    \item Beware, by nature, they cannot work for cross-compilation!
    \item \code{AC_RUN_IFELSE}
    \end{itemize}
  \end{itemize}
\end{frame}

\begin{frame}[fragile]{Writing new tests: {\tt AC\_LINK\_IFELSE}}

\begin{block}{\code{configure.ac}}
{\small
\begin{verbatim}
AC_LINK_IFELSE([AC_LANG_PROGRAM([#include <langinfo.h>],
        [char *codeset = nl_langinfo (CODESET);])],
    [glib_cv_langinfo_codeset=yes],
    [glib_cv_langinfo_codeset=no])
\end{verbatim}}
\end{block}

\begin{block}{Variable in \code{config.log}}
\begin{minted}{console}
$ grep glib_cv_langinfo_codeset config.log
glib_cv_langinfo_codeset=yes
\end{minted}
\end{block}

\end{frame}

\begin{frame}{Printing messages}
  \begin{itemize}
  \item When creating new tests, you may want to show messages,
    warnings, errors, etc.
  \item \code{AC_MSG_CHECKING (feature-description)}
    \begin{itemize}
    \item Notify the user that configure is checking for a
      particular feature.
    \end{itemize}
  \item \code{AC_MSG_RESULT (result-description)}
    \begin{itemize}
    \item Notify the user of the results of a check
    \end{itemize}
  \item \code{AC_MSG_NOTICE (message)}
    \begin{itemize}
    \item Deliver the {\em message} to the user.
    \end{itemize}
  \item \code{AC_MSG_ERROR (error-description, [exit-status = ‘$?/1’])}
    \begin{itemize}
    \item Notify the user of an error that prevents configure from
      completing.
    \end{itemize}
  \item \code{AC_MSG_WARN (problem-description)}
    \begin{itemize}
    \item Notify the configure user of a possible problem.
    \end{itemize}
  \end{itemize}
\end{frame}

\begin{frame}[fragile]{Printing messages: example}

\begin{block}{\code{configure.ac}}
{\small
\begin{verbatim}
AC_MSG_CHECKING([for nl_langinfo])
AC_LINK_IFELSE([AC_LANG_PROGRAM([#include <langinfo.h>],
            [char *codeset = nl_langinfo (CODESET);])],
       [glib_cv_langinfo_codeset=yes],
       [glib_cv_langinfo_codeset=no])
AC_MSG_RESULT([$glib_cv_langinfo_codeset])
\end{verbatim}}
\end{block}

\begin{block}{Execution of \code{./configure}}
\begin{minted}{console}
$ ./configure
[...]
checking for nl_langinfo... yes
[...]
\end{minted}
\end{block}

\end{frame}

\subsection{External software and optional features}

\begin{frame}[fragile]{Using external software}
  \begin{itemize}
  \item When a package uses external software,
    \code{--with-<package>=<arg>} and \code{--without-<package>}
    options are generally offered to control usage of the external
    software.
  \item Implemented using the \code{AC_ARG_WITH} macro.
    \begin{block}{}
{\small
\begin{verbatim}
AC_ARG_WITH (package, help-string,
             [action-if-given], [action-if-not-given])
\end{verbatim}}
    \end{block}
    \begin{itemize}
    \item \code{package} gives the name of the option
    \item \code{help-string} is the help text, visible in
      \code{./configure --help}
    \item \code{action-if-given} is executed when the option is used,
      either positively (\code{--with}) or negatively (\code{--without})
    \item \code{action-if-not-given} is executed when the option is not
      used
    \item \code{<arg>} available as \code{$withval} inside {\em
        action-if-given}, \code{$with_<package>} outside.
    \end{itemize}
  \end{itemize}
\end{frame}

\begin{frame}[fragile]{Package options}

  \begin{itemize}

  \item When a package offers optional features,
    \code{--enable-<feature>} and \code{--disable-<feature>} options
    are generally offered to control the optional feature.
  \item Implemented using the \code{AC_ARG_ENABLE} macro.
\begin{block}{}
{\small
\begin{verbatim}
AC_ARG_ENABLE (feature, help-string,
               [action-if-given], [action-if-not-given])
\end{verbatim}}
\end{block}
\item Usage very similar to the one of \code{AC_ARG_WITH}
\item Value available as \code{$enableval} inside {\em
    action-if-given}, \code{$enable_<feature>} outside.
\end{itemize}

\end{frame}

\begin{frame}[fragile]{Formatting the help string}
  \begin{itemize}
  \item To help formatting the help string, {\em autoconf} provides
    the \code{AS_HELP_STRING} macro
  \item Allows to properly align the different options in the
    \code{./configure --help} output
\begin{block}{}
{\small
\begin{verbatim}
AS_HELP_STRING (left-hand-side, right-hand-side,
      [indent-column = ‘26’], [wrap-column = ‘79’])
\end{verbatim}}
\end{block}
  \end{itemize}
\end{frame}

\begin{frame}[fragile]{{\tt AC\_ARG\_ENABLE} example}

\begin{block}{configure.ac}
{\scriptsize
\begin{verbatim}
AC_ARG_ENABLE([test], AS_HELP_STRING([--enable-test], [Enable tests]),
      [echo "Action if given, val = ${enableval}"],
      [echo "Action if not given"])
echo "enable_test = ${enable_test}"
\end{verbatim}}
\end{block}

\begin{block}{\code{./configure} tests}
\begin{minted}[fontsize=\tiny]{console}
$ ./configure --help
[...]
Optional Features:
[...]
  --enable-test           Enable tests
$ ./configure
[...]
Action if not given
enable_test = 
[...]
$ ./configure --enable-test
[...]
Action if given, val = yes
enable_test = yes
[...]
$ ./configure --disable-test
[...]
Action if given, val = no
enable_test = no
[...]
\end{minted}
\end{block}

\end{frame}

\subsection{Misc}

\begin{frame}{{\tt autoscan}}
  \begin{itemize}
  \item \code{autoscan} is a program provided together with \code{autoconf}
  \item Scans the source tree in the current directory (or the one
    passed as argument)
  \item From that, \code{autoscan}:
    \begin{itemize}
    \item Searches the source files for common portability problems
    \item Checks for incompleteness of the \code{configure.ac} file, if any
    \item Generates \code{configure.scan}, which can be used as a
      preliminary \code{configure.ac}
    \end{itemize}
  \end{itemize}
\end{frame}

\begin{frame}{{\tt AC\_CONFIG\_MACRO\_DIR}}

\end{frame}

\section{Automake advanced}

\begin{frame}{Subdirectories}

SUBDIRS, recursive make, subdir-objects

\end{frame}

\begin{frame}{Conditional compilation}

  Conditional compilation of source,
%%  \url{http://www.gnu.org/software/automake/manual/html_node/Conditional-Sources.html#Conditional-Sources}

  Conditional compilation of programs,
%%  \url{http://www.gnu.org/software/automake/manual/html_node/Conditional-Programs.html#Conditional-Programs}

\end{frame}

\begin{frame}{Building libraries}

%% \url{http://www.gnu.org/software/automake/manual/html_node/A-Library.html#A-Library}
%% \url{http://www.gnu.org/software/automake/manual/html_node/A-Shared-Library.html#A-Shared-Library}

\end{frame}

\begin{frame}{Variables}

LDADD, LDFLAGS

\end{frame}

\begin{frame}{Silent rules}

%% \url{http://www.gnu.org/software/automake/manual/html_node/Silencing-Make.html#Silencing-Make}

\end{frame}

\setuplabframe
{More advanced {\em autotools} usage}
{
  \begin{itemize}
  \item TODO
  \end{itemize}
}

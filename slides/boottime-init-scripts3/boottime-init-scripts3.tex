\begin{frame}
\frametitle{Initramfs for boot time reduction}
Create the smallest initramfs possible, just enough to start the critical
application and then switch to the final root filesystem with
\code{switch_root}:
\begin{itemize}
\item Use a light C library reduced to the minimum,
      {\em uClibc} if you are not yet using it for your root filesystem
\item Reduce BusyBox to the strict minimum.
      You could even do without it and implement \code{/init} in C.
\item Use statically linked applications (less CPU overhead, less
      libraries to load, smaller initramfs if no libraries at all).
      \code{BR2_PREFER_STATIC_LIB} in Buildroot.
\end{itemize}
\end{frame}

\begin{frame}
\frametitle{Statically linked executables: licensing constraints}
\begin{itemize}
\item Statically linked executables are very useful to reduce size
      (especially in small initramfs), and require less work to start.
\item However, the LGPL license in libraries (uClibc, glibc), require
      to leave the user the ability to relink the executable
      with a modified version of the library.
\item Solution to keep static binaries:
      \begin{itemize}
      \item Either provide the executable source code (even
            proprietary), allowing to recompile it with a modified
            version of the library. That's what you do when
            you ship a static BusyBox.
      \item Or also provide a dynamically linked version of the
	    executable (in a separate way), allowing to use another
            library version.
      \end{itemize}
\item References: \\
      {\footnotesize
      \url{http://gnu.org/licenses/gpl-faq.html\#LGPLStaticVsDynamic} \\
      \url{http://gnu.org/copyleft/lesser.html\#section4}
      }
\end{itemize}
\end{frame}

\begin{frame}
\frametitle{Results: after switch to initramfs}
Before:
\begin{center}
    \includegraphics[width=\textwidth]{slides/boottime-init-scripts/timechart-init.pdf}
\end{center}
After:
\begin{center}
    \includegraphics[width=\textwidth]{slides/boottime-init-scripts3/timechart-initramfs.pdf}
\end{center}
\begin{itemize}
\item U-Boot takes more time because the kernel is bigger
\item \code{init} is started earlier because UBI is not initialized yet
      (slow, UBIFS root filesystem)
\end{itemize}
\end{frame}

\begin{frame}
\frametitle{Do not compress your initramfs (1)}
\begin{itemize}
\item If you ship your initramfs inside a compressed kernel image, don't compress
      it \\
      (enable \code{CONFIG_INITRAMFS_COMPRESSION_NONE}).
\item Otherwise, your initramfs data will be compressed twice, and
      the kernel will be slightly bigger and will take a little more time to uncompress.
\end{itemize}
\end{frame}

\begin{frame}
\frametitle{Do not compress your initramfs (2)}
Tests on Linux 3.13-rc4, measuring the penalty of having a
\code{gzip} compressed initramfs in a \code{gzip} compressed kernel. \\
\vfill{}
\scriptsize
Beagle Bone Black (ARM, TI AM3359, 1 GHz)
\begin{tabular}{| l || c | c | c | c | c | }
\hline
Mode & Size & Copy & Uncompress & Total & Diff \\
\hline
No initramfs compression & 4308200 & 451 ms & 945 ms & 5.516 s & \\
Initramfs compression & 4309112 & 455 ms & 947 ms & 5.527 s & + 11 ms \\
\hline
\end{tabular}
\newline\newline
CALAO USB-A9263 (ARM, Atmel AT91SAM9263, 200 MHz)
\begin{tabular}{| l || c | c | c | c | c | }
\hline
Mode & Size & Copy & Uncompress & Total & Diff \\
\hline
No initramfs compression & 3016192 & 4.1047 s & 1.737 s & 8.795 s & \\
Initramfs compression & 3016928 & 4.1050 s & 1.760 s  & 8.813 s & + 18 ms \\
\hline
\end{tabular}
\end{frame}

\begin{frame}
\frametitle{Quick splashscreen display (1)}
Often the first sign of life that you are showing!
\begin{itemize}
\item You could use the \code{fbv} program
      (\url{http://freecode.com/projects/fbv})\\
      to display your splashscreen.
\item On \code{armel}, you can just use our statically compiled
      binary: {\tiny
\url{http://git.free-electrons.com/users/michael-opdenacker/static-binaries/tree/fbv}}
\item However, this is slow:\\
      878 ms on an Atmel AT91SAM9263 system!
\end{itemize}
\end{frame}

\begin{frame}[fragile]
\frametitle{Quick splashscreen display (2)}
\begin{itemize}
\item To do it faster, you can dump the framebuffer contents:\\
      \begin{verbatim}
fbv -d 1 /root/logo.bmp
cp /dev/fb0 /root/logo.fb
lzop -9 /root/logo.fb
      \end{verbatim}
\item And then copy it back as early as possible in an initramfs:
      \begin{verbatim}
lzopcat /root/logo.fb.lzo > /dev/fb0
      \end{verbatim}
\end{itemize}
Results on an Atmel AT91SAM9263 system: \\
\begin{tabular}{| l || c | c | c | }
\hline
& \code{fbv} & plain copy (\code{dd}) & \code{lzopcat} \\
\hline
Time & 878 ms & 54 ms & 52.5 ms\\
\hline
\end{tabular}
\vfill
\footnotesize
\url{http://free-electrons.com/blog/super-fast-linux-splashscreen/}
\end{frame}

\begin{frame}
\frametitle{Animated splashscreen}
Still slow to read and write entire screens. Just draw useful pixels
and even create an animation!
\begin{itemize}
\item Create a simple C program that just animates pixels and simple
      geometric shapes on the framebuffer!
\item Example: {\small \url{http://free-electrons.com/pub/code/fb/anim.c}}.
      On a 400 MHz ARM9 system: starts drawing in 10 ms \\
      size: 24 KB, compiled statically).
\end{itemize}
\end{frame}

\setuplabframe
{Reducing time in init-scripts}
{
\begin{itemize}
\item Simplify user space scripts
\end{itemize}
}

\section{Using Yocto Project - basics}

\subsection{Environment setup}

\begin{frame}
  \frametitle{Environment setup}
  \begin{itemize}
    \item All Poky files are left unchanged when building a custom
      image.
    \item Specific configuration files and build repositories are
      stored in a separate build directory.
    \item A script, \code {oe-init-build-env}, is provided to set up
      the build directory and the environment variables (needed to be
      able to use the \code{bitbake} command for example).
  \end{itemize}
\end{frame}

\begin{frame}
  \frametitle{oe-init-build-env}
  \begin{itemize}
    \item Modifies the environment: has to be sourced!
    \item Adds environment variables, used by the build engine.
    \item Allows you to use commands provided in Poky.
    \item \code{source ./oe-init-build-env [builddir]}
    \item Sets up a basic build directory, named \code{builddir} if it
      is not found. If not provided, the default name is \code{build}.
  \end{itemize}
\end{frame}

\begin{frame}
  \frametitle{Common targets}
  \begin{itemize}
    \item Common targets are listed when sourcing the script:
      \begin{description}[style=nextline]
        \item[core-image-minimal] A small image to boot a device and
          have access to core command line commands and services.
        \item[core-image-sato] Image with Sato support. Sato is a
          GNOME mobile-based user interface.
        \item[meta-toolchain] Includes development headers and
          libraries to develop directly on the target.
        \item[adt-installer] Build the application development toolkit
          installer.
        \item[meta-ide-support] Generates the cross-toolchain. Useful
          when working with the SDK.
      \end{description}
  \end{itemize}
\end{frame}

\begin{frame}
  \frametitle{Exported environment variables}
  \begin{description}[style=nextline]
    \item[BUILDDIR] Absolute path of the build directory.
    \item[BB\_ENV\_EXTRAWHITE] List of environment variables to load
      from the user's environment into BitBake data store.
    \item[PATH] Contains the directories where executable programs are
      located. Absolute paths to \code{scripts/} and
      \code{bitbake/bin/} are prepended.
  \end{description}
\end{frame}

\begin{frame}
  \frametitle{Available commands}
  \begin{description}[style=nextline]
    \item[bitbake] The main build engine command. Used to perform
      tasks on available packages (download, configure, compile\dots).
    \item[bitbake-*] Various specific commands related to the BitBake
      build engine.
    \item[yocto-layer] Command to create a new generic layer.
    \item[yocto-bsp] Command to create a new generic BSP.
  \end{description}
\end{frame}

\begin{frame}
  \frametitle{The \code{build/} directory 1/2}
  \begin{description}[style=nextline]
    \item[conf/] Configuration files. Image specific and layer
      configuration.
    \item[downloads/] Downloaded upstream tarballs of the packages
      used in the builds.
    \item[sstate-cache/] Shared state cache. Used by all builds.
    \item[tmp/] Holds all the build system outputs.
  \end{description}
\end{frame}

\begin{frame}
  \frametitle{The \code{build/} directory 2/2}
  \begin{description}
    \item[tmp/buildstats/] Build statistics for all packages built
      (CPU usage, elapsed time, host, timestamps\dots).
    \item[tmp/deploy/] Final output of the build.
    \item[tmp/deploy/images/] Contains the complete images built by
      the OpenEmbedded build system. These images are used to flash
      the target.
    \item[tmp/work/] Set of specific work directories, split by
      architecture. They are used to unpack, configure and build the
      packages. Contains the patched sources, generated objects and
      logs.
    \item[tmp/sysroots/] Shared libraries and headers used to compile
      packages for the target but also for the host.
  \end{description}
\end{frame}

\subsection{Configuring the build system}

\begin{frame}
  \frametitle{The \code{build/conf/} directory}
  \begin{itemize}
    \item The \code{conf/} directory in the \code{build} one holds 
      build specific configuration.
    \begin{description}
      \item[bblayers.conf] Explicitly list the available layers.
      \item[local.conf] Set up the configuration variables relative to
        the current user for the build. Configuration variables can be
        overridden there.
    \end{description}
  \end{itemize}
\end{frame}

\begin{frame}
  \frametitle{Configuring the build}
  \begin{itemize}
    \item The \code{conf/local.conf} configuration file holds local
    user configuration variables:
    \begin{description}
      \item[BB\_NUMBER\_THREADS] How many tasks BitBake should perform
        in parallel.
      \item[PARALLEL\_MAKE] How many processes should be used when
        compiling.
      \item[MACHINE] The machine the target is built for, e.g.
        \code{beaglebone}.
      \item[PACKAGE\_CLASSES] Packages format (\code{deb}, \code{ipk}
        or \code{rpm}).
    \end{description}
  \end{itemize}
\end{frame}

\subsection{Building an image}

\begin{frame}
  \frametitle{Compilation}
  \begin{itemize}
    \item The compilation is handled by the BitBake {\em build
      engine}.
    \item Usage: \code{bitbake [options] [recipename/target ...]}
    \item To build a target: \code{bitbake [target]}
    \item Building a minimal image: \code{bitbake core-image-minimal}
    \begin{itemize}
      \item This will run a full build for the selected target.
    \end{itemize}
  \end{itemize}
\end{frame}

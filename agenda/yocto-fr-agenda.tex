\documentclass[a4paper,12pt,obeyspaces,spaces,hyphens]{article}

\usepackage{agenda}
\usepackage{colortbl}
\usepackage{xcolor}
\usepackage{palatino}
\usepackage{calc}

\hypersetup{pdftitle={Formation développement Linux embarqué avec Yocto Project et OpenEmbedded},
  pdfauthor={Free Electrons}}

\renewcommand{\arraystretch}{2.0}

\begin{document}

\thispagestyle{fancy}

\setlength{\arrayrulewidth}{0.8pt}

\begin{center}
\LARGE
Formation développement Linux embarqué avec Yocto Project et OpenEmbedded\\
\large
Session de 3 jours
\end{center}
\vspace{1cm}

\small
\newcolumntype{g}{>{\columncolor{fedarkblue}}m{4cm}}
\newcolumntype{h}{>{\columncolor{felightblue}}X}

\arrayrulecolor{lightgray} {
  \setlist[1]{itemsep=-5pt}
  \begin{tabularx}{\textwidth}{|g|h|}
    {\bf Titre} & {\bf Formation développement Linux embarqué avec Yocto Project et OpenEmbedded} \\
    \hline

    {\bf Aperçu} &
    Comprendre l'architecture de Yocto Project \par
    Utilisation pour compiler un système de fichiers et exécuter
    celui-ci sur votre plateforme matérielle \par
    Étendre des recettes ({\em recipes}) existantes et en écrire de nouvelles \par
    Création de {\em layers} \par
    Intégration de votre matériel dans un {\em BSP} \par
    Création d'images sur mesure \par
    Développement applicatif à l'aide d'un SDK basé sur Eclipse \\
    \hline

    {\bf Durée} & {\bf Trois} jours - 24 h (8 h par jour)
    \newline 40\% de présentations et 60\% de travaux pratiques. \\
    \hline

    {\bf Formateur} & Un des ingénieurs mentionnés sur :
    \newline \url{http://free-electrons.com/training/trainers/}\\
    \hline

    {\bf Langue} & Présentations : Français
    \newline Supports : Anglais\\
    \hline

    {\bf Public visé} & Sociétés et ingénieurs intéressés par l'utilisation de Yocto Project
    pour construire leur système Linux embarqué.\\
    \hline

    {\bf Pré-requis} & {\bf Connaissance de Linux embarqué}, sujet couvert par
    notre formation Linux embarqué :
    \newline \url{http://free-electrons.com/training/embedded-linux/} \vspace{1em}
    \newline {\bf Connaissance et pratique des commandes Unix ou
    GNU/Linux}
    \newline Les personnes n'ayant pas ces connaissances doivent se
    former en utilisant nos supports de formations disponibles en ligne :
    \newline (\url{http://free-electrons.com/docs/command-line/} \\
    \hline
  \end{tabularx}

  \begin{tabularx}{\textwidth}{|g|h|}
    {\bf Équipement\newline nécessaire} &
    {\bf Pour les sessions sur site uniquement}
    \newline Le matériel est fourni par Free Electrons durant les
    sessions inter-entreprises
    \begin{itemize}
    \item Projecteur vidéo
    \item Un ordinateur sur chaque bureau (pour une ou deux personnes), avec au
    moins 4 Go de RAM, un processeur au moins équivalent à un Intel Core i5,
    et Ubuntu Linux installé dans une {\bf partition
    dédiée d'au moins 20 Go. L'utilisation de Linux dans une machine virtuelle
    n'est pas supportée}, en raison de problèmes avec la connexion au matériel.
    \item Nous avons besoin d'Ubuntu Desktop 14.04 (32 ou 64 bit, Xubuntu et
    Kubuntu fonctionnent également). Nous ne supportons pas d'autres
    distributions, car nous ne pouvons tester toutes les versions des
    paquets.
    \item {\bf Connexion à Internet} (directe ou par le proxy de l'entreprise).
    \item {\bf Les ordinateurs contenant des données importantes doivent être
    sauvegardés} avant d'être utilisés dans nos sessions. Certains
    participants ont déjà commis des erreurs lors de travaux pratiques
    avec pour conséquence des pertes de données.
    \end{itemize} \\
    \hline

    {\bf Supports} & Versions électroniques et imprimées des
    présentations et travaux pratiques.
    \newline Version électronique des données pour les travaux
    pratiques..\\
    \hline

\end{tabularx}}
\normalsize

\feagendatwocolumn
{Matériel}
{
  La plateforme matérielle utilisée pendant les travaux pratiques de
  cette formation est la carte {\bf BeagleBone Black}, dont voici les
  caractéristiques :

  \begin{itemize}
  \item Un processeur ARM AM335x de Texas Instruments (à base de
    Cortex-A8), avec accélération 3D, etc.
  \item 512 Mo de RAM
  \item 2 Go de stockage eMMC embarqué sur la carte
	\newline(4 Go avec la révision C)
  \item USB hôte et device
  \item Sortie HDMI
  \item Connecteurs à 2 x 46 broches, pour accéder aux UARTs, aux
        bus SPI, aux bus I2C, et à d'autres entrées/sorties du
        processeur.
  \end{itemize}
}
{}
{
  \begin{center}
    \includegraphics[height=5cm]{../slides/beagleboneblack-board/beagleboneblack.png}
  \end{center}
}

\section{1\textsuperscript{er} jour - Matin}

\feagendaonecolumn
{Cours - Introduction aux outils de compilation de systèmes Linux embarqué}
{
  \begin{itemize}
  \item Vue d'ensemble de l'architecture d'un système Linux embarqué
  \item Méthodes pour compiler un système de fichiers
  \item Utilité des outils de compilation
  \end{itemize}
}
\\
\feagendatwocolumn
{Cours - Vue d'ensemble de Yocto Project et du système de référence Poky}
{
  \begin{itemize}
  \item Organisation des sources du projet
  \item Création d'un système de fichiers avec Yocto Project
  \end{itemize}
}
{TP - 1\textsuperscript{ère} compilation avec Yocto Project}
{
  \begin{itemize}
  \item Téléchargement du système de référence Poky
  \item Compilation d'une image système
 \end{itemize}
}

\section{1\textsuperscript{er} Jour - Après-midi}
\feagendatwocolumn
{Cours - Utilisation de Yocto Project - Notions de base}
{
  \begin{itemize}
  \item Structure des fichiers générés
  \item Flasher et installer l'image du système
  \end{itemize}
}
{TP - Flasher et booter}
{
  \begin{itemize}
  \item Flasher et booter l'image du système sur la Beaglebone.
  \end{itemize}
}

\feagendatwocolumn
{Cours - Utilisation de Yocto Project - Utilisation avancée}
{
  \begin{itemize}
  \item Configuration de la compilation
  \item Personnalisation de la sélection de paquetages
  \end{itemize}
}
{TP - Utilisation de NFS et configuration de la compilation}
{
  \begin{itemize}
  \item Configurer la BeagleBone pour démarrer via NFS
  \item Apprendre à utiliser le mécanisme \code{PREFERRED_PROVIDER}
  \end{itemize}
}
\\
\section{2\textsuperscript{ème} jour - Matin}

\feagendatwocolumn
{Cours - Écriture de recettes - Fonctionnalités de base}
{
  \begin{itemize}
  \item Écriture d'une recette minimale
  \item Ajout de dépendances
  \item Organisation du développement avec {\em bitbake}
  \end{itemize}
}
{TP - Ajouter la compilation d'une application}
{
  \begin{itemize}
  \item Création d'une recette pour {\em nInvaders}
  \item Ajout d'{\em nInvaders} à l'image finale
  \end{itemize}
}

\feagendaonecolumn
{Cours - Écriture de recettes - Fonctionnalités avancées}
{
  \begin{itemize}
  \item Extension et redéfinition de recettes
  \item Rajouter des étapes au processus de compilation
  \item Familiarisation avec les classes
  \item Analyse d'exemples
  \item Logs
  \item Mise au point des dépendances
  \end{itemize}
}

\section{2\textsuperscript{ème} jour - Après-midi}

\feagendaonecolumn
{TP - Apprendre à configurer les paquetages}
{
  \begin{itemize}
  \item Extension d'une recette pour ajouter des fichiers de configuration
  \item Utilisation de \code{ROOTFS_POSTPROCESS_COMMAND} pour modifier
        le système de fichier final
  \item Étude des dépendances entre paquetages
  \end{itemize}
}
\feagendatwocolumn
{Cours - Layers}
{
  \begin{itemize}
  \item Ce que sont les {\em layers}
  \item Où trouver les {\em layers}
  \item Création d'un {\em layer}
  \end{itemize}
}
{TP - Écriture d'un layer}
{
  \begin{itemize}
  \item Apprendre à écrire un {\em layer}
  \item Ajouter le {\em layer} à la compilation
  \item Inclure {\em nInvaders} dans le nouveau {\em layer}
  \end{itemize}
}

\section{3\textsuperscript{ème} jour - Matin}

\feagendatwocolumn
{Cours - Écriture d'un BSP}
{
  \begin{itemize}
  \item Extension d'un BSP existant
  \item Ajout d'une nouvelle machine
  \item Chargeurs de démarrage
  \item Linux et la recette linux-yocto
  \item Ajouter un type d'image personnalisé
  \end{itemize}
}
{TP - Mise en oeuvre de modifications du noyau}
{
  \begin{itemize}
  \item Extension de la recette pour le noyau pour ajouter le pilote
        pour le Nunchuk
  \item Configurer le noyau pour compiler le pilote du Nunchuk
  \item Jouer à {\em nInvaders}
  \end{itemize}
}

\section{3\textsuperscript{ème} jour - Après-midi}

\feagendatwocolumn
{Cours - Création d'une image sur mesure}
{
  \begin{itemize}
  \item Écriture d'une recette d'image
  \item Ajouter des utilisateurs et des groupes
  \item Ajouter une configuration personnalisée
  \item Écrire et utiliser des groupes de recettes de paquetages
  \end{itemize}
}
{TP - Création d'une image sur mesure}
{
  \begin{itemize}
  \item Écrire une recette d'image personnalisée
  \item Ajouter {\em nInvaders} à l'image sur mesure
  \end{itemize}
}
\feagendatwocolumn
{Cours - Création et utilisation d'un SDK}
{
  \begin{itemize}
  \item Comprendre l'utilité d'un SDK pour le développeur d'applications
  \item Construire un SDK pour l'image sur mesure
  \end{itemize}
}
{TP - Expérimentations avec le SDK}
{
  \begin{itemize}
  \item Construction d'un SDK
  \item Utilisation du SDK via Eclipse
  \end{itemize}
}

\end{document}


\subchapter{Using a build system, example with Buildroot}{Objectives:
  discover how a build system is used and how it works, with the
  example of the Buildroot build system. Build a Linux system with
  libraries and make it work inside Qemu.}

\section{Setup}

Go to the \code{$HOME/felabs/sysdev/buildroot/} directory,
which already contains some data needed for this lab, including a
kernel image.

\section{Get Buildroot and explore the source code}

The official Buildroot website is available at
\url{http://buildroot.org/}. Download the latest stable 2013.08.x
version which we have tested for this lab. Uncompress the tarball
and go inside the Buildroot source directory.

Several subdirectories or files are visible, the most important ones
are:

\begin{itemize}
\item \code{boot} contains the Makefiles and configuration items
  related to the compilation of common bootloaders (Grub, U-Boot,
  Barebox, etc.)
\item \code{configs} contains a set of predefined configurations,
  similar to the concept of defconfig in the kernel.
\item \code{docs} contains the documentation for Buildroot. You can
  start reading buildroot.html which is the main Buildroot
  documentation;
\item \code{fs} contains the code used to generate the various root
  filesystem image formats
\item \code{linux} contains the Makefile and configuration items
  related to the compilation of the Linux kernel
\item \code{Makefile} is the main Makefile that we will use to use
  Buildroot: everything works through Makefiles in Buildroot;
\item \code{package} is a directory that contains all the Makefiles,
  patches and configuration items to compile the userspace
  applications and libraries of your embedded Linux system. Have a
  look at various subdirectories and see what they contain;
\item \code{system} contains the root filesystem skeleton and the {\em
    device tables} used when a static \code{/dev} is used;
\item \code{toolchain} contains the Makefiles, patches and
  configuration items to generate the cross-compiling toolchain.
\end{itemize}

\section{Configure Buildroot}

In our case, we would like to:

\begin{itemize}
\item Generate an embedded Linux system for ARM;
\item Use an already existing external toolchain instead of having
  Buildroot generating one for us;
\item Integrate Busybox, DirectFB and DirectFB sample applications in
  our embedded Linux system;
\item Integrate the target filesystem into both an ext2 filesystem
  image and a tarball
\end{itemize}

To run the configuration utility of Buildroot, simply run:

\begin{verbatim}
make menuconfig
\end{verbatim}

Set the following options:

\begin{itemize}
\item \code{Target Architecture}: \code{ARM (little endian)}
\item \code{Target Architecture Variant}: \code{arm926t} (we will start booting
      the generated filesystem on an emulated arm9 based system,
      instead of the IGEPv2 board)
\item \code{Toolchain}
  \begin{itemize}
  \item \code{Toolchain type}: \code{External toolchain}
  \item \code{Toolchain}: \code{Custom toolchain}
  \item \code{Toolchain path}: use the toolchain you built:
    \code{/usr/local/xtools/arm-unknown-linux-uclibcgnueabi}
  \item \code{External toolchain C library}: \code{uClibc}
  \item We must tell Buildroot about our toolchain configuration, so:
    enable \code{Toolchain has large file support?}, \code{Toolchain has
    RPC support?}, and \code{Toolchain has C++ support?}.
    Buildroot will check these parameters anyway.
  \end{itemize}
\item \code{System configuration}
  \begin{itemize}
  \item \code{Port to run a getty (login prompt) on}: change \code{ttyS0} to \code{tty1}
  \end{itemize}
\item \code{Target packages}
  \begin{itemize}
  \item Keep \code{BusyBox} (default version) and keep the Busybox
    configuration proposed by Buildroot;
  \item \code{In Graphic libraries and applications (graphic/text)}
    \begin{itemize}
    \item Select \code{directfb}. Buildroot will automatically select the
      necessary dependencies.
    \begin{itemize}
         \item Remove \code{enable touchscreen support}
         \item Select \code{directfb examples}
         \item Select all the DirectFB examples
     \end{itemize}
    \end{itemize}
  \end{itemize}
\item \code{Filesystem images}
  \begin{itemize}
  \item Select \code{ext2/3/4 root filesystem}
  \item Select \code{tar the root filesystem}
  \end{itemize}
\end{itemize}

Exit the menuconfig interface. Your configuration has now been saved
to the \code{.config} file.

\section{Generate the embedded Linux system}

Just run:

\begin{verbatim}
make
\end{verbatim}

Buildroot will first create a small environment with the external
toolchain, then download, extract, configure, compile and install each
component of the embedded system.

All the compilation has taken place in the \code{output/} subdirectory. Let's
explore its contents:

\begin{itemize}

\item \code{build}, is the directory in which each component built by
  Buildroot is extract, and where the build actually takes place

\item \code{host}, is the directory where Buildroot installs some
  components for the host. As Buildroot doesn't want to depend on too
  many things installed in the developer machines, it installs some
  tools needed to compile the packages for the target. In our case it
  installed pkg-config (since the version of the host may be ancient)
  and tools to generate the root filesystem image (genext2fs,
  makedevs, fakeroot)

\item \code{images}, which contains the final images produced by
  Buildroot. In our case it's just an ext2 filesystem image and a
  tarball of the filesystem, but depending on the Buildroot
  configuration, there could also be a kernel image or a bootloader
  image. This is where we find \code{rootfs.tar} and
  \code{rootfs.ext2}, which are respectively the tarball and the ext2
  image of the generated root filesystem.

\item \code{staging}, which contains the “build” space of the target
  system. All the target libraries, with headers, documentation. It
  also contains the system headers and the C library, which in our
  case have been copied from the cross-compiling toolchain.

\item \code{target}, is the target root filesystem. All applications
  and libraries, usually stripped, are installed in this
  directory. However, it cannot be used directly as the root
  filesystem, as all the device files are missing: it is not possible
  to create them without being root, and Buildroot has a policy of not
  running anything as root.

\item \code{toolchain}, is the location where the toolchain is
  built. However, in our configuration, we re-used an existing
  toolchain, so this directory contains almost nothing.

\end{itemize}

\section{Run the generated system}

If you didn't do it in the previous lab, install QEMU emulator for non
x86 targets:

\begin{verbatim}
sudo apt-get install qemu-kvm-extras
\end{verbatim}

We will use the kernel image in \code{data} and the filesystem image
generated by Buildroot in the ext2 format to boot the generated system
in QEMU. We start by using a QEMU emulated ARM board with display
support, allowing to test graphical applications relying on the
DirectFB library. Later, we will be able move to a real board if your
hardware also has a graphical display.

Execute the \code{run_qemu} script, which contains what's needed
to boot the system in QEMU.

Log in (\code{root} account, no password), and run demo programs:

\begin{verbatim}
df_andi
df_dok
df_fire
...
\end{verbatim}

\section{Tests on the IGEPv2 board}

If you have graphical displays to connect your IGEPv2 board to (LCD
display with DVI-D or HDMI input, TI Pico Projector...), and the
corresponding cables (HDMI to DVI-D or HDMI to HDMI), you can test
your root filesystem on the board.

You can skip to the next section if you don't have a display.

Here's what you can do:

\begin{itemize}
\item Continue to use the same kernel as in the previous labs
\item Extract the tar file for the generated root filesystem in a
\code{nfsroot} directory. Make your board boot on this directory. 
\item In the root filesystem, you may need to edit \code{etc/inittab},
  and add a new line similar to the one running a \code{getty} in
  \code{ttyO2}, but on \code{tty0} instead (Caution: this is a
  {\em zero} character this time.
\item Connect your board to an DVI-D or HDMI display
\item Rebuild your kernel with the below configuration settings
(Currently broken but expect to work in 3.13)
\begin{verbatim}
CONFIG_OMAP2_DSS=y
CONFIG_FB_OMAP2=y
CONFIG_DISPLAY_PANEL_DPI=y
CONFIG_DISPLAY_CONNECTOR_DVI=y
CONFIG_FRAMEBUFFER_CONSOLE_DETECT_PRIMARY=y
CONFIG_LOGO=y
CONFIG_USB_EHCI_HCD=y
CONFIG_USB_EHCI_TT_NEWSCHED=y
CONFIG_USB_EHCI_HCD_OMAP=y
CONFIG_USB_OHCI_HCD=y
CONFIG_USB_OHCI_HCD_OMAP3=y
\end{verbatim}

Note that the USB settings are only needed if you want to connect a USB
mouse and keyboard (behind a USB 2.0 hub), to interact with the
framebuffer console and with the DirectFB programs.

To make your life simpler, we also included a complete Linux
configuration file in the \code{data} folder in the lab directory.

\item Add the following settings to the boot arguments:
  \begin{itemize}
  \item \code{console=tty0}\\(allows you to have both a framebuffer
    and serial console)
  \item \code{vram=12M}\\(video RAM)
  \item \code{omapfb.mode=dvi:640x480MR-16@60}\\(example for the
    Pico Projector. You may first try to do without this parameter,
    and then specify a mode that your monitor supports).
  \end{itemize}
\end{itemize}

\section{Going further}

\begin{itemize}

\item Add dropbear (SSH server and client) to the list of packages
  built by Buildroot, add the network emulation in QEMU (see the
  \code{../thirdparty/run_qemu} script for an example), and log to
  your target system in QEMU using a ssh client on your development
  workstation. Hint: you will have to set a non-empty password for the
  root account on your target for this to work.

\item Add a new package in Buildroot for the GNU Gtypist game. Read
  the Buildroot documentation to see how to add a new
  package. Finally, add this package to your target system, compile it
  and run it in QEMU.

\end{itemize}

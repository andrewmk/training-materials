\subchapter{Input interface}{Objective: make the I2C device available
to user space using the input subsystem.}

After this lab, you will be able to:

\begin{itemize}
\item Expose device events to user space through an input interface,
      using the kernel based polling API for input devices
      (kernel space perspective).
\item Handle registration and allocation failures in a clean
      way.
\item Get more familiar with the usage of the input interface
      (user space perspective).
\end{itemize}

\section{Add polled input device support to the kernel}

The nunchuk doesn't have interrupts to notify the I2C master that
its state has changed. Therefore, the only way to access device data
and detect changes is to regularly poll its registers, using the input
polling API described in the lectures.

Rebuild your kernel with static support for polled input device support
(\code{CONFIG_INPUT_POLLDEV=y}). With the default configuration, this
feature is available as a module, which is less convenient.

Update and reboot your kernel.

\section{Register an input interface}

The first thing to do is to add an input device to the system. Here are
the steps to do it:

\begin{itemize}
\item Declare a pointer to an \code{input_polled_dev} structure in the
      \code{probe} routine. You can call it \code{polled_input}.
      You can't use a global variable because your driver needs to be
      able to support multiple devices.
\item Allocate such a structure in the same function, using the
      \code{input_allocate_polled_device()} function.
\item Also declare a pointer to an \code{input_dev} structure. You can
      call it \code{input}. We won't need to allocate it, because it
      is already part of the \code{input_polled_dev} structure,
      and allocated at the same time.
      We will use this as a shortcut to keep the code simple.
\item Still in the \code{probe()} function, add the input device to
      the system by calling \code{input_register_polled_device()};
\end{itemize}

At this stage, first make sure that your module compiles well (add
missing headers if needed).

\section{Handling probe failures}

In the code that you created, make sure that you handle failure
situations properly.

\begin{itemize}
\item Of course, test return values properly and log
      the causes of errors.
\item If the call to \code{input_register_polled_device()} fails,
      you must also free the \code{input_polled_dev} structure
      before returning an error. If you don't do that, you will create
      memory leaks in the kernel. In the general case, failure to
      release things that have been allocated or registered before
      can prevent you from reloading a module.
\end{itemize}

To implement this correctly without duplicating or creating ugly code,
it's recommended to use \code{goto} statements.

See {\em Chapter 7: Centralized exiting of functions} in
\code{Documentation/CodingStyle} for useful guidelines and an example.
Implement this in your driver.

\section{Implement the remove() function}

We now need to release the resources allocated and registered in the
\code{probe()} routine.

However, this is not trivial in the way we implemented the
\code{probe()} routine. As we have to support multiple devices, we chose
not to use global variables, and as a consequence, we can't use such
global variables to release the corresponding resources.

This raises a very important aspect of the device model: the need to
keep pointers between {\em physical} devices (devices as handled by the
physical bus, I2C in our case) and {\em logical} devices (devices handled by subsystems,
like the input subsystem in our case).

This way, when the \code{remove()} routine is called (typically if the
bus detects the removal of a device), we can find out which logical
device to unregister. Conversely, when we have an event on the logical
side (such as opening or closing an input device for the first time),
we can find out which i2c slave this corresponds to, to do the specific
things with the hardware.

This need is typically implemented by creating a {\em private} data
structure to manage our device and implement such pointers between
the physical and logical worlds.

Add the below definition to your code:

\begin{verbatim}
struct nunchuk_dev {
        struct input_polled_dev *polled_input;
        struct i2c_client *i2c_client;
};
\end{verbatim}

Now, in your \code{probe()} routine, declare an instance of
this structure:

\begin{verbatim}
        struct nunchuk_dev *nunchuk;
\end{verbatim}

Then allocate one such instead for each new device:

\sourcecode{labs/kernel-i2c-input-interface/private-data-alloc.c}

Note that we haven't seen kernel memory allocator routines and flags
yet.

All we can say for the moment is that with \code{devm_} functions,
each allocation or registration
is attached to a device structure. When a device or a module is removed,
all such allocations or registrations are automatically undone. This
allows to greatly simplify driver code.

We haven't explained the \code{dev_*} logging routines yet either
(they are basically used to tell which device a given log message is
associated to).  For the moment, just use the above code. You will get
all the details later.

Now implement the pointers:

\sourcecode{labs/kernel-i2c-input-interface/device-pointers.c}

Make sure you add this code before registering the input device. You
don't want to enable a device with incomplete information or when it is
not completely yet (there could be race conditions).

With all this in place, you now have everything you need to implement
the \code{remove()} routine. Look at the I2C and input slides for
examples, or directly find your examples in the Linux kernel code!

Recompile your module, and load it and remove it multiple times, to
make sure that everything is properly registered and unregistered.

\section{Add proper input device registration information}

We actually need to add more information to the input structure before
registering it. That's why we are getting the below warnings:

\begin{verbatim}
input: Unspecified device as /devices/virtual/input/input0
\end{verbatim}

Add the below lines of code (still before device registration, of
course):

\sourcecode{labs/kernel-i2c-input-interface/input-device-attributes.c}

Recompile and reload your driver. You should now see in the kernel log
that the \code{Unspecified device} type is replaced by
\code{Wii Nunchuck} and that the physical path of the device is reported
too.

\section{Implement the polling routine}

It's time to implement the routine which will poll the nunchuk registers
at a regular interval.

Create a \code{nunchuck_poll()} function with the right prototype (find
it by looking at the definition of the \code{input_polled_dev} structure.

First, add lines retrieving the I2C physical device from the
\code{input_polled_dev} structure. That's where you will need your
private \code{nunchuk} structure.

Now that you have a handle on the I2C physical device, you can move the
code reading the nunchuk registers to this function. You can
remove the double reading of the device state, as the polling function
will make periodic reads anyway \footnote{During the move, you will have
to handle communication errors in a slightly different way, as the
\code{nunchuk_poll()} routine has a \code{void} type. When the function
reading registers fails, you can use a \code{return;} statement instead
of \code{return value;}}.

At the end of the polling routine, the last thing to do is post the events
and notify the \code{input} core. Assuming that \code{polled_input} is the
name of the \code{input_polled_dev} parameter of your polling routine:

\sourcecode{labs/kernel-i2c-input-interface/input-notification.c}

Now, back to the \code{probe()} function, the last thing to do
is to declare the new polling function (see the slides if you forgot
about the details) and specify a polling interval of 50 ms.

You can now make sure that your code compiles and loads successfully.

\section{Testing your input interface}

Testing an input device is easy with the \code{evtest} application
that is included in the root filesystem. Just run:

\begin{verbatim}
evtest
\end{verbatim}

The application will show you all the available input devices, and will let
you choose the one you are interested in (make sure you type a choice,
\code{0} by default, and do not just type \code{[Enter]}). You can also
type \code{evtest /dev/input/event0} right away.

Press the various buttons and see that the corresponding events are
reported by \code{evtest}.

\section{Going further}

If you complete your lab before the others, you can add support for the
nunchuk joystick coordinates.

Another thing you can do then is add support for the nunchuk accelerometer
coordinates.
